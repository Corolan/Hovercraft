\documentclass[a4paper, 12pt, twoside, openright]{report}

\usepackage[top=2cm, bottom=2cm, left=2.5cm, right=2.5cm]{geometry}
\usepackage[T1]{fontenc}
\usepackage[polish=nohyphenation]{hyphsubst} % zapobiega dzieleniu wyrazów
\usepackage[polish]{babel}
\usepackage[utf8]{inputenc}
\usepackage{lmodern}
\usepackage{graphicx}
\usepackage[nottoc]{tocbibind}
%\usepackage{showframe}
\selectlanguage{polish}
\usepackage{subfiles}

\title{Praca magisterska}
\date{2019-10-25}
\author{Piotr Gajcy}

\begin{document}

\pagenumbering{gobble}
% \maketitle
\begin{titlepage}
	\centering
	{\large \textbf{WOJSKOWA AKADEMIA TECHNICZNA}} \\
	{\large im. Jarosława Dąbrowskiego} \\
	{\large \textbf{WYDZIAŁ ELEKTRONIKI}} \\
	\noindent\rule{\textwidth}{4pt} \\
	\vspace{10pt}
	\includegraphics[width=0.30\textwidth]{../../../logoWAT.png}\par\vspace{1cm}
	\vspace{20pt}
	{\LARGE \textbf{PRACA DYPLOMOWA}} \\
	\vspace{20pt}
	**Temat pracy** \\ 
	\vspace{10pt}
	inż Piotr Gajcy, syn Zbigniewa	\\ 
	\vspace{10pt}
	{\large \textbf{ELEKTRONIKA I TELEKOMUNIKACJA}} \\ 
	\vspace{10pt}
	**Specjalność** \\ 
	\vspace{10pt}
	Niestacjonarne studia drugiego stopnia \\ 
	\vspace{10pt}
	**Dane promotora** \\ 
	
	\vspace*{\fill}
	{\large \textbf{WARSZAWA 2019}}
	
\end{titlepage}
\newpage
  
Tekst testowy przed spisem treści.
%\pagenumbering{roman}
  
\pagenumbering{arabic}
\tableofcontents
\newpage
    
\chapter*{Wstęp}
\addcontentsline{toc}{chapter}{Wstęp}
Hello World!

\chapter{Teoria zagadnienia}
\subfile{../rozdzialy/rozdzial1.tex} 
\chapter{Model poduszkowca}
\subfile{../rozdzialy/rozdzial2.tex}
\chapter{Konstrukcja urządzenia}
\subfile{../rozdzialy/rozdzial3.tex}
\chapter{Program sterujący}
Bo ściany.
\chapter{Komunikacja bezprzewodowa}
Bluetooth.
\chapter{Testy urządzenia}
Wszystko działa!
\chapter*{Podsumowanie}
\addcontentsline{toc}{chapter}{Podsumowanie}

\addtocontents{toc}{\protect\vspace{20pt}}

\listoffigures
\listoftables

\bibliographystyle{plain}
\bibliography{bibliografia}


\end{document}
